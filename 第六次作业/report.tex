\documentclass[UTF8]{ctexart}
\usepackage{geometry, CJKutf8}
\geometry{margin=1.5cm, vmargin={0pt,1cm}}
\setlength{\topmargin}{-1cm}
\setlength{\paperheight}{29.7cm}
\setlength{\textheight}{25.3cm}

% useful packages.
\usepackage{amsfonts}
\usepackage{amsmath}
\usepackage{amssymb}
\usepackage{amsthm}
\usepackage{enumerate}
\usepackage{graphicx}
\usepackage{multicol}
\usepackage{fancyhdr}
\usepackage{layout}
\usepackage{listings}
\usepackage{float, caption}

\lstset{
    basicstyle=\ttfamily, basewidth=0.5em
}

% some common command
\newcommand{\dif}{\mathrm{d}}
\newcommand{\avg}[1]{\left\langle #1 \right\rangle}
\newcommand{\difFrac}[2]{\frac{\dif #1}{\dif #2}}
\newcommand{\pdfFrac}[2]{\frac{\partial #1}{\partial #2}}
\newcommand{\OFL}{\mathrm{OFL}}
\newcommand{\UFL}{\mathrm{UFL}}
\newcommand{\fl}{\mathrm{fl}}
\newcommand{\op}{\odot}
\newcommand{\Eabs}{E_{\mathrm{abs}}}
\newcommand{\Erel}{E_{\mathrm{rel}}}

\begin{document}

\pagestyle{fancy}
\fancyhead{}
\lhead{顾善元, 3230104463}
\chead{数据结构与算法第六次作业}
\rhead{Nov.9rd, 2024}

\section{删除操作概述}
`BinarySearchTree`类的删除操作在处理节点的删除的同时保持AVL树的平衡。实现步骤如下:

\begin{enumerate}
    \item 如果要删除的节点没有子节点,直接将其从树中移除。
    \item 如果节点有一个子节点,用其子节点替换该节点。
    \item 如果节点有两个子节点,找到中序后继,将其值复制到要删除的节点,然后删除中序后继。
\end{enumerate}

\section{平衡树}
保持AVL属性,删除操作执行以下步骤:

\begin{enumerate}
    \item 更新受影响节点的高度。
    \item 重新计算每个节点的平衡因子。
    \item 执行旋转以重新平衡树。
\end{enumerate}

旋转基于左子树和右子树高度之差进行。可能的旋转包括:

\begin{itemize}
    \item 左左和左右情况执行右旋转。
    \item 右右和右左情况执行左旋转。
\end{itemize}

\end{document}

%%% Local Variables: 
%%% mode: latex
%%% TeX-master: t
%%% End: 
