\documentclass{article}
\usepackage{ctex} % 引入中文包
\usepackage{amsmath}
\usepackage{graphicx}

\title{堆排序报告}
\author{顾善元}
\date{\Nov 30th}

\begin{document}

\maketitle

\section{设计与实现}
\subsection{HeapSort.h}
\texttt{HeapSort.h}堆排序算法基于二叉堆数据结构,通过构建最大堆或最小堆来实现排序。算法分为两个主要部分:构建最大堆和从堆中提取元素进行排序。

\subsection{测试用例}
测试用例包括随机序列、有序序列、逆序序列和部分元素重复的序列。

\section{测试流程}
1. 生成四种类型的测试序列:随机序列、有序序列、逆序序列和部分元素重复序列,每个序列的长度均为1000000。
2. 对每种序列使用自定义堆排序算法和`std::sort_heap()`函数进行排序。
3. 记录并比较两种方法的排序时间。

\section{结果与对比}
以下表格比较了自定义堆排序和\texttt{std::sort\_heap()}在不同类型序列上的执行时间:

\begin{center}
\begin{tabular}{|c|c|c|}
\hline
\textbf{序列类型} & \textbf{我的堆排序时间 (s)} & \textbf{std::sort\_heap 时间 (ms)} \\
\hline
随机序列 & \textbf{1664} & \textbf{478} \\
有序序列 & \textbf{1283} & \textbf{301} \\
逆序序列 & \textbf{1969} & \textbf{430} \\
部分重复元素序列 & \textbf{1808} & \textbf{531} \\
\hline
\end{tabular}
\end{center}

\end{document}