\documentclass[UTF8]{ctexart}
\usepackage{geometry, CJKutf8}
\geometry{margin=1.5cm, vmargin={0pt,1cm}}
\setlength{\topmargin}{-1cm}
\setlength{\paperheight}{29.7cm}
\setlength{\textheight}{25.3cm}

% useful packages.
\usepackage{amsfonts}
\usepackage{amsmath}
\usepackage{amssymb}
\usepackage{amsthm}
\usepackage{enumerate}
\usepackage{graphicx}
\usepackage{multicol}
\usepackage{fancyhdr}
\usepackage{layout}
\usepackage{listings}
\usepackage{float, caption}

\lstset{
    basicstyle=\ttfamily, basewidth=0.5em
}

% some common command
\newcommand{\dif}{\mathrm{d}}
\newcommand{\avg}[1]{\left\langle #1 \right\rangle}
\newcommand{\difFrac}[2]{\frac{\dif #1}{\dif #2}}
\newcommand{\pdfFrac}[2]{\frac{\partial #1}{\partial #2}}
\newcommand{\OFL}{\mathrm{OFL}}
\newcommand{\UFL}{\mathrm{UFL}}
\newcommand{\fl}{\mathrm{fl}}
\newcommand{\op}{\odot}
\newcommand{\Eabs}{E_{\mathrm{abs}}}
\newcommand{\Erel}{E_{\mathrm{rel}}}

\begin{document}

\pagestyle{fancy}
\fancyhead{}
\lhead{顾善元, 3230104463}
\chead{数据结构与算法第五次作业}
\rhead{Nov.2rd, 2024}

\section{remove的设计思路}
我设计修改后的remove函数在提高删除操作的效率,避免不必要的节点内容复制和递归删除种情况下都能正确运行。先递归查找要删除的元素  
,再处理找到要删除的元素的情况,最后用detachMin函数递归地找到并删除子树中的最小节点。

\section{测试的结果}
正确性验证:
节点是否被正确删除:检查树中是否不再包含被删除的节点。
树的结构是否保持:确保删除操作后,树仍然保持二叉搜索树的性质,即左子节点的值小于父节点,右子节点的值大于父节点。
异常处理:确保在尝试删除不存在的节点时,能够正确抛出和捕获异常。
性能分析:
时间复杂度:分析remove操作的时间复杂度,理想情况下应该是O(log n),其中n是树中节点的数量。
空间复杂度:检查是否有不必要的空间使用,如递归调用栈的深度。
边界条件:
空树:确保在空树上调用remove函数时能够正确处理。
单节点树:测试在只有单个节点的树上调用remove函数。
最大值和最小值:测试删除树中的最大值和最小值。
  


\end{document}

%%% Local Variables: 
%%% mode: latex
%%% TeX-master: t
%%% End: 
