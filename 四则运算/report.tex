\documentclass{article}
\usepackage[utf8]{inputenc}
\usepackage{ctex}

\title{中缀表达式求值程序报告}
\author{顾善元}
\date{12/11}

\begin{document}

\maketitle

\section{引言}
中缀表达式求值程序的设计和实现。程序支持加、减、乘、除四则运算,并能处理包含括号的复杂表达式。同时,程序能够判断输入表达式的合法性,并在不合法的情况下输出“ILLEGAL”。

\section{设计思路}

\subsection{需求分析}
程序需满足以下需求:
\begin{itemize}
  \item 支持加(+)、减(-)、乘(*)、除(/)四则运算。
  \item 支持括号(())的使用。
  \item 支持有限位小数运算。
  \item 能够判断表达式的合法性,并在非法时输出“ILLEGAL”。
\end{itemize}

\subsection{模块划分}
程序分为两个主要模块:
\begin{itemize}
  \item \texttt{expression\_evaluator.h}:声明求值相关的函数和异常类。
  \item \texttt{expression\_evaluator.cpp}:实现表达式求值和合法性检查的具体逻辑。
\end{itemize}

\subsection{算法选择}
程序采用两个栈(一个用于数字,一个用于运算符)来处理运算符优先级和括号。处理复杂的表达式求值问题。

\section{实现细节}

\subsection{异常类定义}
定义了两个自定义异常类,用于处理除以零和非法表达式的情况。

\subsection{合法性检查}
实现函数\texttt{isLegalExpression},检查表达式中的括号是否匹配、是否以运算符开头或结尾、以及是否有连续的运算符。

\subsection{表达式求值}
实现函数\texttt{evaluateExpression},使用栈来处理运算符优先级和括号,计算表达式的值。

\section{测试案例与结果分析}

\subsection{测试案例}
设计了多个测试案例,包括合法和非法的表达式,以及包含小数和括号的复杂表达式。

\subsection{结果分析}
程序能够正确处理所有测试案例,并在遇到非法表达式时输出“ILLEGAL”。对于合法的表达式,程序能够准确计算出结果。

\section{结论}
本程序成功实现了中缀表达式的求值功能,并能够判断表达式的合法性。程序具有良好的健壮性和准确性。

\end{document}